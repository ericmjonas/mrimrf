\documentclass{beamer}
\usetheme{Warsaw}

\mode<presentation>

\title{Bayesian Phase Unwrapping with Factor Graphs}
\author{Eric Jonas}
\date{May 12, 2009}
\institute[6.556]{MIT Department of Brain and Cognitive Sciences}


\begin{document}

\begin{frame}
\maketitle
\end{frame}

\section{Phase Unwrapping for MR}


\section{Factor Graphs and Markov Random Fields}
\begin{frame}
\frametitle{Markov Random Fields and Factor Graphs}
Markov Random Field: A particular type of probabilistic model
Factor Graph: a particular language for describing graphical models
\cite{Kschischang01}

\end{frame}

\begin{frame} 
\frametitle{Factor Graphs for Low-Level Vision}
Properties of image MRFs
large number of verticies
O(1) (constant local) connectivity
\end{frame}

\begin{frame} 
\frametitle{MRFs for Phase Unwrapping}
Two views on our MRF
\end{frame}

\section{Inference in MRFs}

\begin{frame}
  \frametitle{Markov-Chain Monte Carlo}
  Set up a state space so that the expectation is the target distribution
  Used in situations where you know $\pi^\ast(x)$ but not $\pi(x)$

\end{frame}

\begin{frame}
  \frametitle{Metropolis Hastings}
  forward backward jacobian 
\end{frame}

\begin{frame}
  \frametitle{Gibbs Sampling}
  like MH but along an axis

\end{frame}

\begin{frame}
  \frametitle{Tempering} 
  Like Simulated Annealing
\end{frame}

\begin{frame}
  \frametitle{Swendsen-Wang}
  Work Through
\end{frame}

\begin{frame}
  \frametitle{MRFs and Parallelism}
\end{frame}

\section{Performance}

\subsection{Synthetic Data}
\begin{frame}
  \frametitle{2-D Synthetic Data}
\end{frame}

\begin{frame}
  \frametitle{3-D Synthetic Data}
\end{frame}

\subsection{Phantoms}

\subsection{Actual Data}
\begin{frame}
  \frametitle{Div and Audrey}
\end{frame}

\section{Methods Comparison}
\begin{frame}
  \frametitle{PRELUDE}
\end{frame}

\section{Concluson and Future Directions}
\begin{frame}
  \frametitle{Where to now?}
  Exact sampling using Systematic Stochsatic Search
  Better neighborhood connectivity / likelihood? 
  GPU implementation 
  Better visualization of posterior?
\end{frame}

\end{document}
