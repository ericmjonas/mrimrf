

My project will begin by initially attempting to perform 3-d phase
unwrapping using Markov Random Fields and comparing this approach with
existing tools, such as the PRELUDE package from FSL.

I'm going to be using Brenden Frey's 2006 IEEE Trans Biomed. Image. paper
on the topic as a starting point. Frey's paper uses a form of iterated
conditional modes to find the MAP estimate for the MRF. 

I plan on using a combination of Markov-chain Monte Carlo approximate
sampling algorithms to solve the MRFs, which I feel will: 
  1. prove easier to implement
  2. be less computationally efficient but much more parallelizable
  3. be more amenable to extension, by decoupling the ``model'' we're 
solving (sampling from) from the inference method. 

I may also attempt to implement a form of Systematic Stochastic
Search, a scheme for producing _exact_ samples from MRFs that I have
worked on with some collaborators and will be presented at this
month's AI Stats conference.

By decoupling the model from the inference step, I hope to be able to
extend Frey's original model in several ways: 
   1. 3D vs 2D
   2. create hyper-priors on the latent smoothness potential 
   3. explore advantages in moving from a lattice to a more  densely-connected
MRF. 

I have already acquired data from Div and Audrey for initial experimentation. 
This week I plan on beginning playing with that data, and constructing the
initial MRF for inference. 
